\documentclass[11pt]{scrartcl}
\usepackage{dirtytalk}

\title{Philosophy of Science (720A04), VT2019, Exam}
\subtitle{Teacher: Valdi Ingthorsson (\texttt{valdi.ingthorsson@gmail.com)}}
\author{Maximilian Pfundstein}

\begin{document}

\maketitle

\tableofcontents

\newpage

\section{Assignments}

Please answer the following questions in essay format. Answers are graded on the basis of completeness and depth, not length, but as a rule of thumb 1 - 1,5 (+/- 20\%) pages per question should be about right. I award up to 8 points per answer. 50\% for Pass, 75\% for Pass with distinction. Deadline is 24 May at 17:00. Hand-in via Lisam.

\subsection{Hypothetico-Deductive Method}

\textbf{Question:} What is the hypothetico-deductive method, and what are its strengths and weaknesses with respect to its ability to verify and/or falsify hypotheses? Do you think science revolves around the use of the H-D method?

\bigbreak

\textbf{Answer:} The problem with applying inductivism is well described in Understanding Philosophy of Science from James Ladyman \cite[p. 40]{ladyman}: \say{Since it is logically possible that any regularity will fail to hold in the future, the only basis we have for inductive inference is the belief that the future will resemble the past.}

So it seems that it is not possible to use the true principles of induction, but science needs a way to circumvent this problem, otherwise it would be impossible to build any chain of validity. The resources for checking out all possibilities and outcomes are either too demanding or not possible at all. That's when the Hypothetico-Deductive methods comes in, as it suggests a way of handling this problem. It does not solve the problem, but rather shows a way how to deal with a theory or hypothesis.

We will use the following example throughout this answer to have a vivid hypothesis. The hypothesis is, that all swans are black. Depending on where one looks up information about the Hypothetico-Deductive method, the methods consist of three or four steps:

\begin{itemize}
  \item Hypothesis
  \item Prediction
  \item Test of Predictions
\end{itemize}

The literature relying on four steps include, that for finding a hypothesis, previous knowledge and experience must be used and therefore results in another step. So from the observations a hypothesis is being phrased. In our example the hypothesis is, that all swans are black. In this three-step explanation we assume that the is at least a strong hint towards the hypothesis, but the method itself holds also for random hypothesis if the steps are correctly and thoroughly applied.

The hypothesis needs one crucial property which is that is falsifiable. It's important that this property is falsifiable by science in general (maybe in the future) as we otherwise leave no room for our hypothesis being invalid and thus making the method itself inapplicable for science. In our example, we could phrase a statement that if true, falsifies the hypothesis: \say{If we see at least one non-black swan, the hypothesis is wrong}.

The next step is that we use the hypothesis (in science mostly a theory) to make predictions. Either for using the predictions directly in a scientific context or to further fortify the hypothesis. When stating a hypothesis or theory it makes sense to try to falsify it, because if we fail doing so, we have strong assumption that the hypothesis is true. Note that this method will never yield in evidence as long as we don't observe every possible outcome.

These trials in falsifying the hypothesis belong to the test of predictions and important so strengthen or weaken the hypothesis. One important note is that this hypothetico-deductive method does not say anything about the usefulness of a theory. For instance we know that Newtons laws of mechanics are wrong from a scientific point of view, but that the predictions for low velocities are so close to the better model of general relativity, that they are still incredible useful and wildly used.

This example shows that this method has some flaws as it does not include a measure of usefulness. Also the assumption that the hypothesis must be falsifiable is not always given in applied science. If we take religions for an example, they mostly do not phrase a logical statement for falsifying them but are still accepted throughout the world. This sounds like this method is relatively useless, but I think that in a scientific context this method is a really good basement for getting started and missing aspects like the mentioned usefulness are left for us humans to add.

To conclude this answer I say that I agree that science in general evolves around this method but that there are many exceptions to this and the given hypothesis are not always clearly stated. If we look at research papers they rarely following this method in an obvious way. Therefore I think that sciences evolves around the method, but does not strictly follow it.

\subsection{Scientific Paradigms}

\textbf{Question:} What is a scientific paradigm and how do they influence scientific practice? Is it good or bad that science is guided by paradigms? Do you think the programme you have chosen is schooling you into a particular programme?

\bigbreak

\textbf{Answer:} There exists no exact definition for the scientific paradigm but according to Kuhn there are two applications of scientific paradigms that can we identified: The paradigm as \textit{disciplinary matrix} and the paradigm as \textit{exemplar}. I will briefly discuss both of them.

\say{A disciplinary matrix is a set of answers to such questions that are learned by scientists in the course of the education that prepares them for research, [...]} \cite[p. 98]{ladyman}. This is anticipated as a base to work with and is passed on over time. In general the matrix exists of more or less explicit assumptions and aspects and there is a consensus that that some explanations and ways of tackling problems are preferred over others. This distinguishes this paradigm from a theory. James Ladyman calls this specific skills, because that's what results from the application over time, \textit{tacit knowledge} which is for example the skill to focus a telescope or experimental skills.

The second application is being called \textit{exemplar} and tries to provide the skillset and models for handling future problems. For obtaining this skillset, well-known problems are given to the inexperienced scientist where they learn how to approach a new problem and learn the problem-solving process. This way they can use their obtained skillset, which is not the knowledge about the problem itself, but how to work in this specific domain of science.

The term \textit{normal science} refers to most science as it describes working within the field of a well-established paradigm according to Kuhn \cite[p. 100]{ladyman}. This utilises the paradigm and pushes towards the suggested direction. If the paradigm is checked thoroughly, this drastically reduces the amount of possibilities in the field of search but is also susceptible in missing out solutions that are contrary to the currently used paradigm. Usually this tends to less questioning. Kuhn is critical to Poppers falsificationism \cite[p. 101]{ladyman} as Popper implies that scientists should abandon neglected theories. I agree with Kuhn being critical as the mentioned example from Newtons theories about motion are the perfect case where just because a theory or paradigm is proven wrong it's not implied that it's useless.

In case that a paradigm has too many serious flaws it happens that the paradigm is given up upon in favour of another. We have seen this in the past, albeit rarely. This is called \textit{Paradigm Shift}. Last we should also be aware that paradigms are sociological and psychological constructs. So I agree with James Ladyman when the says that paradigms are intellectual properties of rules (even though they do not have something like a right of ownership).

To conclude the question if paradigms are good or bad, I want to emphasise that I don't favour one side. The benefits are that paradigms mirror our humans perception of the world, trying to simplify things to make them comprehendible. We need that to function properly, but it also raises the issue missing out on important things. Paradigms are also in some way social constructs and I would say that paradigms do not add these social constructs but rather are just their shape in the scientific community.

As we study Statistics we have this paradigms as well, the most common one probable the difference between the Bayesian and the Frequentist approach about observations and true believes. While the Bayesian way assumes that the data is given and the model parameters are to be set, the Frequentists assume that the model parameters are fixed and produce the data. Both have their benefits and are easier to apply in different areas.


\subsection{Falsifiability}

\textbf{Question:} What does it mean for a scientific hypothesis to be falsifiable, and: (i) why is it good that they are falsifiable, and (ii) why is even better that they can be falsified in many different ways?

\bigbreak

\textbf{Answer:} TBA.

\subsection{Theory-Dependency of Observation}

\textbf{Question:} In what way are observations theory-dependent, and why does that challenge the idea that hypotheses are generated inductively from observations?

\bigbreak

\textbf{Answer:} TBA.

\subsection{Difference between Natural and Human Sciences}

\textbf{Question:} What is the difference between the natural and the human sciences according to Ingthorsson? Include a reflection on what Ingthorsson says about the nature of the phenomena that the natural and human sciences study, and relate to what that nature implies about differences in method.

\bigbreak

\textbf{Answer:} TBA.

\subsection{Difference Between Science and Pseudo-Science}

\textbf{Question:} What is the difference between science and pseudo-science according to Sven-Ove Hansson, and why should we care?

\bigbreak

\textbf{Answer:} TBA.

\subsection{Being a Scientific Realist}

\textbf{Question:} What does it involve to be a scientific realist, and what reasons can we have for adopting that position? Do you think those reasons are convincing, and do think it would make any difference for you to take a realist or anti-realist approach to research in your discipline?

\bigbreak

\textbf{Answer:} TBA.

\section{Sources}

\bibliography{bib/literature}
\bibliographystyle{ieeetr}

\end{document}
