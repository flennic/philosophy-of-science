\documentclass[11pt]{scrartcl}

\title{Philosophy of Science (720A04), VT2019, Exam}
\subtitle{Teacher: Valdi Ingthorsson (\texttt{valdi.ingthorsson@gmail.com)}}
\author{Maximilian Pfundstein}

\begin{document}

\maketitle

\tableofcontents

\newpage

\section{Assignments}

Please answer the following questions in essay format. Answers are graded on the basis of completeness and depth, not length, but as a rule of thumb 1–1,5 (+/- 20\%) pages per question should be about right. I award up to 8 points per answer. 50\% for Pass, 75\% for Pass with distinction. Deadline is 24 May at 17:00. Hand-in via Lisam.

\subsection{Hypothetico-Deductive Method}

\textbf{Question:} What is the hypothetico-deductive method, and what are its strengths and weaknesses with respect to its ability to verify and/or falsify hypotheses? Do you think science revolves around the use of the H-D method?

\bigbreak

\textbf{Answer:} TBA.

\subsection{Scientific Paradigms}

\textbf{Question:} What is a scientific paradigm and how do they influence scientific practice? Is it good or bad that science is guided by paradigms? Do you think the programme you have chosen is schooling you into a particular programme?

\bigbreak

\textbf{Answer:} TBA.

\subsection{Falsifiability}

\textbf{Question:} What does it mean for a scientific hypothesis to be falsifiable, and: (i) why is it good that they are falsifiable, and (ii) why is even better that they can be falsified in many different ways?

\bigbreak

\textbf{Answer:} TBA.

\subsection{Theory-Dependency of Observation}

\textbf{Question:} In what way are observations theory-dependent, and why does that challenge the idea that hypotheses are generated inductively from observations?

\bigbreak

\textbf{Answer:} TBA.

\subsection{Difference between Natural and Human Sciences}

\textbf{Question:} What is the difference between the natural and the human sciences according to Ingthorsson? Include a reflection on what Ingthorsson says about the nature of the phenomena that the natural and human sciences study, and relate to what that nature implies about differences in method.

\bigbreak

\textbf{Answer:} TBA.

\subsection{Difference Between Science and Pseudo-Science}

\textbf{Question:} What is the difference between science and pseudo-science according to Sven-Ove Hansson, and why should we care?

\bigbreak

\textbf{Answer:} TBA.

\subsection{Being a Scientific Realist}

\textbf{Question:} What does it involve to be a scientific realist, and what reasons can we have for adopting that position? Do you think those reasons are convincing, and do think it would make any difference for you to take a realist or anti-realist approach to research in your discipline?

\bigbreak

\textbf{Answer:} TBA.

\section{Sources}

\end{document}
