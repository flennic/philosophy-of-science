\documentclass[11pt]{scrartcl}

\usepackage{dirtytalk}
\usepackage[breaklinks, colorlinks=true]{hyperref}

\usepackage{palatino}

\hypersetup{
	citecolor={blue},
	linkcolor={red},
}

\title{Philosophy of Science (720A04), VT2019, Exam}
\subtitle{Teacher: Valdi Ingthorsson (\texttt{valdi.ingthorsson@gmail.com)}}
\author{Maximilian Pfundstein}

\begin{document}

\maketitle

\tableofcontents

\newpage

\section{Assignments}

Please answer the following questions in essay format. Answers are graded on the basis of completeness and depth, not length, but as a rule of thumb 1 - 1,5 (+/- 20\%) pages per question should be about right. I award up to 8 points per answer. 50\% for Pass, 75\% for Pass with distinction. Deadline is 24 May at 17:00. Hand-in via Lisam.

\subsection{Hypothetico-Deductive Method}

\textbf{Question:} What is the hypothetico-deductive method, and what are its strengths and weaknesses with respect to its ability to verify and/or falsify hypotheses? Do you think science revolves around the use of the H-D method?

\bigbreak

\textbf{Answer:} In general it can be said that inductivism is preferred over deductivism as it leaves no doubts in the chain of logic once applied. Still deductivism is sometimes preferred over inductivism as inductivism is infeasible to apply under some circumstances. The problem with applying inductivism is well described in Understanding Philosophy of Science from Ladyman \cite[p. 40]{ladyman}: \say{Since it is logically possible that any regularity will fail to hold in the future, the only basis we have for inductive inference is the belief that the future will resemble the past.}

So it seems that it is not possible to use the true principles of induction, but science needs a way to circumvent this problem, otherwise it would be impossible to build any chain of validity. The resources for checking out all possibilities and outcomes are either too demanding or not possible at all. That is when the Hypothetico-Deductive methods comes in, as it shapes a way of handling this problem. It does not solve the problem directly, but rather shows how to deal with a theory or hypothesis to still make it useful and applicable in science.

As it is easier to describe this method with an example, we will use the following logical statement throughout this answer to have a more comprehensible hypothesis: \say{All swans are white}.

Depending on where one looks up information about the Hypothetico-Deductive method, the methods consist of three or four steps.

\begin{itemize}
  \item Hypothesis
  \item Prediction
  \item Test of Predictions
\end{itemize}

Literature relying on four steps include, that for finding a hypothesis, previous knowledge and experience must be used and therefore results in an additional first step. Thus a hypothesis is being phrased by observing a pattern that could be used to make predictions. In our example the hypothesis is, that all swans are white, presumably because the observer has only seen white swans so far. It is assumed that there is at least a strong hint towards the hypothesis and no obvious contradiction against it. Note that the method itself will also hold for a random hypothesis if the steps are correctly and thoroughly applied even if this leads to a larger rejection rate.

The presented hypothesis needs one crucial property which is that it must be falsifiable. It is important that this property is falsifiable by science in general (maybe in the future) as we otherwise leave no room for our hypothesis being invalid and thus making the method itself inapplicable for science. In our example, we could phrase a statement that if true, falsifies the hypothesis: \say{If we see at least one non-white swan, the hypothesis is wrong}.

The next step is that we use the hypothesis (in science mostly a theory) to make predictions and then either use them directly in a scientific context or further fortify the hypothesis. The latter one is usually done first to strengthen the hypothesis. When stating a hypothesis or theory it makes sense to try to falsify it, because if we fail doing so, we have strong assumption that the hypothesis is actually true. Note that this method will never yield in evidence as long as we do not observe every possible outcome.

These attempts in falsifying the hypothesis belong to the last step which is testing if the predictions hold. Keep in mind that this Hypothetico-Deductive method does not say anything about the usefulness of a theory or hypothesis. For instance we know that Newtons laws of motions are wrong from a scientific point of view, but that the predictions for low velocities are so close to the actual model of general relativity, that they are still incredible useful and widely used.

This shows that this method has some flaws as it does not include a measure of usefulness for instance. Also the assumption that the hypothesis must be falsifiable is not always given in applied science. If we take religions for an example, they mostly do not phrase a logical statement for falsifying them but are still accepted throughout the world. This tends towards assuming that  this method is not that useful, but I think that in a scientific context this method is a really good basement for getting started in creating a theory. Missing aspects like the mentioned usefulness are left for us humans to evaluate.

To conclude this answer I agree that science in general evolves around this method but that there are many exceptions to this and the given hypothesis are not always clearly stated. If we look at research papers they rarely follow this method in a direct way. Therefore I think that sciences evolves around the method, but does not strictly follow it.

\subsection{Scientific Paradigms}

\textbf{Question:} What is a scientific paradigm and how do they influence scientific practice? Is it good or bad that science is guided by paradigms? Do you think the programme you have chosen is schooling you into a particular programme?

\bigbreak

\textbf{Answer:} There exists no exact definition for the scientific paradigm but according to Kuhn there are two applications of scientific paradigms that can be identified: The paradigm as a \textit{disciplinary matrix} and the paradigm as an \textit{exemplar} \cite{ladyman}. I will briefly discuss both of them.

\say{A disciplinary matrix is a set of answers to such questions that are learned by scientists in the course of the education that prepares them for research, [...]} \cite[p. 98]{ladyman}. So the disciplinary matrix is anticipated as a base to work with and is passed on over generations of scientists. The matrix consists of more or less explicit assumptions and aspects and a consensus exists that some explanations and ways of tackling problems are preferred over others. This distinguishes this paradigm from a theory. Ladyman calls these specific skills \textit{tacit knowledge}, because that is what results from applying this paradigm over time, which is for example the skill to focus a telescope or experimental skills \cite[p. 99]{ladyman}.

The second application is being called \textit{exemplar} and tries to provide the skillset and models for handling future problems. For obtaining this skillset, well-known problems are given to the inexperienced scientists where they learn how to approach a new problem and the problem-solving process itself. This way they can use their obtained skillset, which is not necessarily the knowledge about the previous problems themselves, to make scientific progress in this specific domain of science.

The term \textit{normal science} refers to most science as it describes working within the field of a well-established paradigm according to Kuhn \cite[p. 100]{ladyman}. If the paradigm was checked thoroughly, this drastically reduces the amount of possibilities in the field of research. This utilisation of the paradigm can result in faster progress because usually this tends to less questioning. Nevertheless this is also susceptible in missing out solutions that are contrary to the currently accepted paradigm.

Sometimes we find a paradigm which is more useful then the previous one, so what do we do with the old one? Kuhn is critical to Poppers falsificationism \cite[p. 101]{ladyman} as Popper implies that scientists should abandon neglected theories and paradigms. I agree with Kuhn being critical as the earlier mentioned example about Newtons theory of motion are the perfect example of a theory proven wrong and still being useful.

So in case that a paradigm has too many serious flaws it happens that the paradigm is given up upon in favour of another. We have seen this in the past, albeit rarely. This is called \textit{Paradigm Shift}. We should also be aware that paradigms are sociological and psychological constructs. So I agree with Ladyman when he says that paradigms are intellectual properties of rules (even though they do not have something like a right of ownership).

To conclude the question if paradigms are good or bad, I want to emphasise that I do not favour one side. The benefits are that paradigms mirror our human perception of the world, trying to simplify things to make them comprehendible. We rely on simplifications to function as humans, but it also raises the issue missing out on important aspects. Paradigms itself are not intended to be social constructs by default, but it lies in the human nature to shape them in a way that they are.

As we study Statistics we have these paradigms as well, the most common one is probable the difference between the Bayesian and the Frequentist approach about observations and true beliefes. While the Bayesian way assumes that the data is given and the model parameters are to be set, the Frequentists assume that the model parameters are fixed and produce the data. Both have their benefits and are easier to apply in different areas.

\subsection{Falsifiability}

\textbf{Question:} What does it mean for a scientific hypothesis to be falsifiable, and: (i) why is it good that they are falsifiable, and (ii) why is even better that they can be falsified in many different ways?

\bigbreak

\textbf{Answer:} For a scientific hypothesis being falsifiable means that there is at least one logical statement that refutes the hypothesis. This statement must be scientifically falsifiable. We will talk more about what is to be considered scientific in one of the following paragraphs \footnote{According to Popper.}. Ladyman gives the following examples of statements that are not falsifiable \cite[p. 69]{ladyman}:

\begin{itemize}
  \item Either it is raining or it is not raining.
  \item God has no cause.
  \item All bachelors are unmarried.
  \item It is logically possible that space is infinite.
  \item Human beings have free will.
\end{itemize}

The issue with these statements is that the number of observations will never be enough to either accept or refute any of the hypothesis. According to him a hypothesis is scientific, if it is falsifiable. The amount of supporting observations is irrelevant. This leads us to the question why it actually is good for a scientific hypothesis to be falsifiable.

If a scientific hypothesis is falsifiable it adds value to the hypothesis because it shows us on the one hand that the scientific knowledge is justified and and the other hand it shows its limits. If we look at the statements above we actually see that the more general they are, the less value they add. The statement \say{Either it is raining or it is not raining} is always true, non-falsifiable and thus not scientific.

The reason to value a hypothesis based on falsification instead of the amount of \textit{positive} observed events is that in general it is very easy \say{[...] to accumulate positive instances which support some theory, especially when the theory is so general in its claims that its seems not to rule anything out} \cite[p. 66]{ladyman}. This means that these hypotheses do not rule anything out and thus are of lesser value.

This directly brings us to the next question why it is even better if a scientific hypothesis can be falsified in many different ways. The reason for that is, that is adds more value to the hypothesis. Popper calls these \say{novel-predictions} \cite[p. 68]{ladyman} as they make predictions about so far undiscovered phenomenas. One of the very best examples of this is the recent confirmation of gravitational waves with \say{significance greater than $5.1\sigma$} \cite{ligo}. This hypothesis had and still has a remarkable value as there are many observations which could have shown the contrary. Even now it is not yet 100 percent confirmed. As we have learned in an earlier question, this will actually be really challenging, but science follows the paradigm that at a specific significance level something is being considered as confirmed.

As we can clearly see, it was a great commitment of Einstein to predict gravitational waves. Other theories and hypothesis lack this commitment. Ladyman summarises: \say{It is this commitment to their theories that Popper thinks is unscientific. In fact, he demands of scientists that they specify in advance under what experimental conditions they would give up their most basic assumptions.} \cite[p. 71]{ladyman}

So to conclude the answer to the question why it is even better if a scientific hypothesis is falsifiable in many ways I want to add that mostly newer hypotheses, which are more specific than previous ones, tend to be easier falsifiable and thus have greater value. This lies in the nature that they have to make even better predictions to hold \cite[p. 73]{ladyman} and I fully agree about that with Popper. Again all this does not tell if something is \say{meaningful} \cite[p. 72]{ladyman} or useful, as I have phrased it before. Religions or beliefs \textit{can} have a large influence, positively as negatively, even if not fulfilling the requirements of being falsifiable.

\subsection{Theory-Dependency of Observation}

\textbf{Question:} In what way are observations theory-dependent, and why does that challenge the idea that hypotheses are generated inductively from observations?

\bigbreak

\textbf{Answer:} Observations rely on something between themselves and our consciousness and whatever we assume it is, is prone to modifications we are not aware of. Those can either emerge from subjective perception or lie in nature itself. There are different reasons why observations are theory-dependent, so let us take a look at them \footnote{Source used: http://www.herinst.org/envcrisis/science/method/theory.html}.

\begin{itemize}
  \item When we observe an event or conduct an experiment we have to use our senses for example to hear or see the outcome. Not talking about the limitations of our senses itself there is still a lot of processing that goes on before the signal reaches our brain and eventually forms information. For instance our retina has a specific resolution and it uses that to map different wave lengths to stimulate different nerves that use electrical signals to actually deliver the information. \textit{So observations do not only depend on our senses}.
  \item Sometimes we have an intermediate layer like words, numbers or pictures. While most of them are quite clear in their meaning, some of them are not. If a screen says the signal is \textit{strong} we have our own beliefs what that actually means, we use our pre-understanding of the word \textit{strong}. Another example is if a colour is shown we have our own experiences how that looks like (or not if one is colour-blind). \textit{So observations assume pre-knowledge.}
  \item The experiment itself is just a construct we create to observe and we assume that what instruments show us, more specific the measurement itself, is an indicator for what we want to measure. Some of these require pre-knowledge, the understanding of a specific theory or interpretations. \textit{So observations are supported by experiments.}
  \end{itemize}
  
 Taking this into account the problem with the idea that hypotheses are generated inductively from observations is that induction assumes that its steps, the observations, are valid. If they are not they cannot be used for inference and the theory-dependence actually adds this problem to our observations. To be critical there are further problems with induction which are summarised by Ladyman: \say{Since it is logically possible that any regularity will fail to hold in the future, the only basis we have for inductive inference is the belief that the future will resemble the past. But that the future will resemble the past is something that is only justified by past experience, which is to say, by induction, and the justification of induction is precisely what is in question. Hence, we have no justification for our inductive practices and they are the product of animal instinct and habit rather than reason} \cite[p. 40]{ladyman}.
 
These two issues, namely the \textit{validity of observations} and \textit{justification of practice}, challenge induction and require further investigation.

\subsection{Difference between Natural and Human Sciences}

\textbf{Question:} What is the difference between the natural and the human sciences according to Ingthorsson? Include a reflection on what Ingthorsson says about the nature of the phenomena that the natural and human sciences study, and relate to what that nature implies about differences in method.

\bigbreak

\textbf{Answer:} Before delving into the details of the difference between natural and human sciences according to Ingthorsson we should have a brief look at the problems and issues that raised the discussion in the first place. Chapter \say{The value of the human sciences} \cite[p. 29]{ingtho} addresses these issues which arise from the assumption that natural science is superior to human science as natural science helped mankind to achieve accomplishments like reaching the moon or curing diseases. Ingthorsson points out that progress in law, social structures or social structures has been made as well but due to the difficulty in assigning value to these achievements no credits to human science is given.

To understand the common accusations directed towards human science we first have to understand what natural and human science are and what their differences are. It is important to emphasise that these accusations basically evolve around the hypothesis that natural science are more valuable, as they can make accurate predictions, and their methods more sophisticated. We will see later on that, according to Ingthorson, \say{[...] natural sciences have been allotted the easier task, [...]} \cite[p. 40]{ingtho}.

Ingthorsson says that natural science is the study of \say{unconscious physical matter in all its forms.} \cite[p. 40]{ingtho} which he calls \textit{the merely physical}. This includes the field of medicine as long as it is not about psychological topics e.g. psychosomatic disorders because they suffer from the same problem as human sciences: \say{lack of decisive evidence and strict laws that can give accurate predictions and/or treatments} \cite[p. 28]{ingtho}. To abridge and spare the details we can say that natural science is about unconscious subject and their discovery. I favour including phenomenas which are also related to non-matter like electromagnetism as well, in contrary to Ingthorson, as they clearly belong to the domain of natural sciences as well (compare page 28 in \cite{ingtho}).

Human sciences focus on animate beings, so Ingthorsson refers to them as \textit{meaningful phenomenas} \cite[p. 27]{ingtho}. They have to use different methods for observing their surroundings and are less likely to make accurate predictions, which lies in the nature of things they study. This needs to be emphasised because we will see later, that this is the main fact that distinguishes human sciences from natural sciences. Ingthorsson also discusses the difference between \textit{Objective Reality} and \textit{Objective Knowledge}. The main point is that \say{social constructions are objectively real} \cite[p. 37]{ingtho} and thus worth being studies, even if accurate predictions cannot be made. An example Ingthorsson mentions are nations, which are social constructs and very real, even if they cannot exists independent from a (collective) mind.

Concluding from this, the difference between natural and human sciences is, that natural science is all about inanimate, observable, empiric, useable and predictable things while human sciences are about social constructs, understanding behaviour and not necessarily predictable subjects. This brings us to the reflection of the nature of phenomena which will later be used by Ingthorson to deduct his conclusions.

Ingthorsson mentions an example of salt which dissolves in water \cite[p. 32]{ingtho}. This observation is repeatable and can be generalised so that predictions can be concluded from previous experiments. He claims that this actually makes natural sciences the easier of the two: \say{The simple reason being that physical nature is lawful and predictable; it offers laws on a silver plate} \cite[p. 32]{ingtho}. Salt is mostly not unreliable and does not have it's own will. Humans will act differently in the same situation. They will also act differently if the same experiments is done multiple times. This actually makes it more difficult to analyse these kind of experiments. How would the natural sciences react, if they had this unstable behaviour? Ingthorsson brings up a rather funny question: \say{You ever seen a Goth grain of salt, or a Punk electron?} \cite[p. 33]{ingtho}. This shows that the underlying nature of phenomena both sciences investigate, are different. This somehow automatically leads to the assumption that both should sciences should use and rely on different methods.

In conclusion Ingthorsson points out that it is unfounded that \say{human sciences are methodologically retarded} \cite[p. 40]{ingtho}. Knowledge about social behaviour should not be based on methods from natural science, but should have their own. Actually they are supposed to have their own methods as they deal with different phenomenas. One is lawful while the other one is not. He also states that he is not saying that human sciences are at their peak of flourishing, but deserve more \say{respect to their subject of matter} \cite[p. 41]{ingtho}. Both must not be ignorant but utilise their mutual strengths and learn from is other instead of disparage each other.
\bigbreak


\subsection{Difference Between Science and Pseudo-Science}

\textbf{Question:} What is the difference between science and pseudo-science according to Sven-Ove Hansson, and why should we care?

\bigbreak

\textbf{Answer:} First we will take a glance at the meaning of the word science from a philosophical perspective before we will look at the meaning of the word \textit{pseudo}. Then we will draw the demarcation between science and pseudo-science.

The definition of science goes into two distinct directions. The former focuses on descriptive contents and how the term is used. The latter focuses on the normative element and tries to explain the fundamental meaning of the term. Most philosophers choose the latter definition. Thus the term science mostly refers to natural science. In addition to the obvious fields of biology, physics and chemistry science also entails political economy and sociology while studies of literature and history are not. This distinction is specific to the English language, therefore the term science will denote natural sciences as well as human sciences within the scope of this answer.

There exist several ways of defining the negation of science where all of them have a different definition, especially about the intentions of the user. The term user is used as the word \textit{scientist} is only fully applicable in science and not in most of the negations of science. The term \textit{unscientific} implies some contradiction to science itself as it pretends to use scientific methods whereas the term \textit{non-scientific} does not try to make this claim. Therefore the term pseudo science belongs to the domain of unscientific methods, due to the fact that pseudo scientists count themselves to the domain of \textit{real} science. People would probably agree that there \textit{doings} are non-scientific but usually people would not agree that their doings are unscientific, thus pseudo-scientific. The former could be a subjective analysis or conclusion from experience where the actor is aware of the non-scientific nature of his experiment. The latter describes people that use methods, which are not scientific, but think that they are. Examples for this would be people supporting the flat-earth theory, astrology or homeopathy.

Taking these definitions into account, it seems easy to draw a line between science and pseudo science. The difference lies in the methods used, whereas the scientific ones are supported by scientific standards and the pseudo scientific are not. Still it seems that there is an obvious contradiction which is not yet solved, because no-one would refer to themselves as a pseudo scientist. So let us conclude this answer by shedding some light on this matter.

Hansson says about pseudo science that \say{it is part of a doctrine whose major proponents try to create the impression that it represents the most reliable knowledge on its subject matter} \footnote{https://plato.stanford.edu/entries/pseudo-science/\#PsePse}. This implies that people supporting on of the doctrines assume a different base of arguments. This could be taking the bible as a fundament or any other subjective perception of the world. Following the chain of arguments some of their theories might actually hold, but are still invalid as they fundamentally assume something that is not scientific. One can have a long discussion about what it actually means for something to be scientific, but usually people agree that it envelopes concepts like falsification, corrections and comprehensibility.

Concluding from this, we should care and distinct between science and pseudo science as the latter is simply wrong and pollutes scientific research. Unfortunately it's not always that easy to see the difference, especially when one is not involved in the scientific field as has to assume that the fundamentals are correct, at least to some extend. Pseudo scientists use this issue to spread their theories. Most of the pseudo scientists try to enforce their theories using further pseudo scientific (e.g. subjective) methods which actually expose them. It seems easy to distinguish between both, but on matters where scientific methods are unsure or tend to belong to the field of human sciences, it becomes harder to draw the line. Therefore I think scientists should be aware of this problem and should not be afraid to unmask pseudo science.



\subsection{Being a Scientific Realist}

\textbf{Question:} What does it involve to be a scientific realist, and what reasons can we have for adopting that position? Do you think those reasons are convincing, and do think it would make any difference for you to take a realist or anti-realist approach to research in your discipline?

\bigbreak

\textbf{Answer:} Scientific Realism is \say{about whatever is described by our best scientific theories} \footnote{Source: https://plato.stanford.edu/entries/scientific-realism/\#ThreDimeRealComm}. It is divided into three dimensions which will be explained each on their own. The dimensions are the \textit{metaphysical dimension}, also called \textit{ontological dimensions}, the \textit{semantic dimension} and the \textit{epistemological dimension.}

The metaphysical dimension commits to the existence of the world explained by science while being mind-independent. Note that some similarities to the natural sciences can be drawn as it focuses on things which can be used for predictions. It's basically the contrary to \textit{Idealism} which states that reality or what humans perceive as reality is fundamentally mental.

The semantical dimensions says that realism is basically a literal interpretation of scientific statements about the world. This means a so called \textit{face value} is assigned to each claim about observables and henceforth truth values are elaborated stating if the observable is true or false.

The last dimension, the epistemological dimension, constitutes knowledge of the world which means that there exist reasons that the theoretical claims made by the theory are true.

Putting the pieces together we obtain a perception of the world, which is called scientific realism. It expects science to make progress in the long-term which unfolds in either better theories or the answering of more and more questions - or both. So being a scientific realist means to commit to these paradigms. One of the reasons for scientific realism is the \textit{no miracles argument}. It says that by observing that scientific methods are successful they constitute the best theory we have about the world. It is the only way to describe the world without miracles and as their predictions are adequately precise it's the best view of the world we can have \footnote{Some consider this wrong due to the \textit{Base Rate Fallacy}. More about can be read here: http://www.fallacyfiles.org/baserate.html}. A point against scientific realism is the \textit{pessimistic conduction}. It argues that theories in the past, which were considered to be true, turned out to be wrong, even though back then they have been ostensible empirically proven. Also sometimes theories have unobservable terms or claims which therefore is replaced later on by a better theory.

There are more pros and cons, as well as many more examples, still this short introduction should provide us with the information we need to answer the question for reasons one can have for adapting the one or other position. Speaking for me neither the pros nor the cons accomplish to convince me. That lies in the nature of how I personally evaluate things in my life which is a rather pragmatic approach. This simply means I use what is useful. As an example I think Newtons laws of motion are useful, so I use them, at least in some domains. If I have to built a GPS system, I fall back to use Einsteins laws of general relativity. I refrain to commit to anything that is neither falsifiable nor provable as it adds no value. I count religions to those. My believe is that they are extraordinary useless for anything. Of course they have an influence from a social perspective and can be used from either cheering up individuals and providing a thrive in life up controlling masses and nations. As this questions aims at if it makes a difference for me to take one specific side, the answer is no. Actually I consider myself to have no affinity towards any of the positions, as both have flaws in their argumentation and it will not help me in my research. If I were forced to pick a side, it would be pro scientific realism, as it provided the more useful framework for me and I don't have a problem with lying theories aside that are proven wrong. That's the whole point of having a \textit{theory}.



\section{Sources}

\bibliography{bib/literature}
\bibliographystyle{ieeetr}

\end{document}
