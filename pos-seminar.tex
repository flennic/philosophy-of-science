\documentclass[11pt]{scrartcl}

\title{Philosophy of Science (720A04), VT2019, Seminar}
\subtitle{Teacher: Valdi Ingthorsson (\texttt{valdi.ingthorsson@gmail.com)}}
\author{Maximilian Pfundstein}

\begin{document}

\maketitle

\tableofcontents

\newpage

\section{Assignments}

Please answer the following questions in essay format. Answers are graded on the basis of completeness and depth, not length, but as a rule of thumb 1–1,5 (+/- 20\%) pages per question should be about right. I award up to 8 points per answer. 50\% for Pass, 75\% for Pass with distinction. Deadline is 24 May at 17:00. Hand-in via Lisam.

\subsection{What are the researchers studying?}

\textbf{Notes:}
\begin{itemize}
  \item Do the studied phenomena belong to the domain of the natural or the human sciences?
  \item Are they objectively measurable or subjectively evaluated?
  \item Are the methods used well suited to study the phenomena they are interested in?
\end{itemize}

\textbf{Answers:}
\begin{itemize}
  \item Natural Science as it is not mainly about emotions or Psychology.
  \item It seems that the numbers itself are objectively measured but the conclusions lack the objectivism.
  \item This is really hard to judge as a person not coming from the field.
\end{itemize}

\bigbreak

\subsection{Do you find that the articles reveal any particular view about what science is all about?}

\textbf{Notes:}
\begin{itemize}
  \item Empirical/theoretical
  \item Testing of hypotheses/descriptive/exploratory
  \item Positivistic/falsificationist/hermeneutic
  \item Theory-dependent/independent
  \item Objective measurement/subjective interpretation
\end{itemize}

\textbf{Answers:}
\begin{itemize}
  \item The results are more empirical as they mostly observe correlations but don't theoretically derive them.
  \item The taken approach is positivistic as it is based on natural  phenomenas consuming knowledge obtained through observations. It's not classified as falsifiable as there can be no statement on its own showing the contrary (it has to be at least statistically relevant).
  \item It seems the measurement itself is objective, at least to a reasonable degree, but the interpretations are subjective as they heavily use words like \textit{seem}, \textit{might}, \textit{could} or \textit{probably} which does not indicate an objective interpretation despite the fact that they compare their results with other applied researches. The reason for that is that, depending on \textit{when} they looked at other results, the authors biased themselves.
\end{itemize}

\bigbreak

\subsection{Do the articles reveal the authors’ views about knowledge?}

\textbf{Notes:}
\begin{itemize}
  \item Data = fact
  \item Insight
  \item True justified belief / practically useful ideas/beliefs that are neither true or false
\end{itemize}

\textbf{Answers:}
\begin{itemize}
  \item They solely rely on their data even though their peer group is small and might not be statistically significant.
  \item This paper focuses more on finding correlation, maybe even causality, but not on explanation.
  \item It can be seen that they rely on other papers, software and hardware, so they follow the general accepted academic way of doing research (building on top of results of others, assuming they worked correctly, at least as long as their is no strong evidence against that).
\end{itemize}

\bigbreak

\subsection{Are the authors self-critical to their arguments/hypotheses/ conclusions?}

\textbf{Notes:}
\begin{itemize}
  \item Discuss sources of bias/confounders?
  \item Discuss alternative explanations?
  \item Are conclusions justified by their results?
\end{itemize}


\textbf{Answers:}
\begin{itemize}
  \item As already mentioned their interpretations are subjective, maybe also their measurements (but this is really hard to judge when not coming from this area of research). Bias might have been implied spoiling the results with looking at other researches too early. As we don't know when this might have taken place, we cannot really make a strong statement here.
  \item They are not really self-critical. They use blurred interpretations so they always have a way to withdraw their statement.
  \item Their conclusions are not in direct contrary to the observed data.
\end{itemize}

\bigbreak

\subsection{Discuss whether the authors’ preunderstanding is reflected in the text, or do we have to speculate about it}

\textbf{Notes:}
\begin{itemize}
  \item Do they refer to earlier research
  \item Do they refer to research that contradict them?
  \item Do they introduce relevant theories?
  \item Do they justify the validity of the method?
\end{itemize}

\textbf{Answers:}
\begin{itemize}
  \item Yes the refer to earlier research a lot.
  \item Not really, but they mention that earlier methods showed opposing results and significance.
  \item They have guesses how their found results can be obtained, but they have, at least in my opinion, no relevant theory.
  \item Yes, with their obtained data. It's to be judged if this taken approach is sufficient.
\end{itemize}

\bigbreak

\end{document}
